\documentclass{beamer}
\usetheme{Warsaw}
\usecolortheme{beaver}
\usepackage{polski}

\usepackage{multimedia}
\usepackage{media9}

\title{Prezentacja do Projektu}
\subtitle{Eksploracja i wizualizacja danych}
\author{Michał Brodacki, s32038}
\institute{Polsko--Japońska Akademia Technik Komputerowych}
\date{11 Grudnia 2023}

\begin{document}
	\begin{frame}
		\titlepage
	\end{frame}
	\begin{frame}{Spis Treści}
		\tableofcontents % Slajd ze spisem treści
	\end{frame}
	
	\section{Cel i Dane}
	\subsection{Cel}
	\begin{frame}
		\begin{block}{Cel Projektu}
		Celem projektu jest rozpoznanie na podstawie danych pozycyjnych koszykarza oddającego rzut, czy będzie on liczony jako $3$--punktowy, czy jako $2$--punktowy.
		\end{block}

	\end{frame}
	\subsection{Dane}
\begin{frame}
	\frametitle{Dane}
Do projektu wykorzystam dane NBA 2023 Player Shot Dataset dostępne pod linkiem: 
\textit{\url{https://www.kaggle.com/datasets/dhavalrupapara/nba-2023-player-shot-dataset/?select=2_james_harden_shot_chart_2023.csv}}. Zawierają one informacje na temat sytuacji rzutowych trzech koszykarzy występujących w \textbf{NBA}.
\end{frame}

	\subsection{Wstępna ocena danych}
\begin{frame}
	\frametitle{Wstępna ocena danych}
	Tu napisać coś o danych (których kolumn używamy i czy może się powieść) i dać kawałek tabelki
	
\end{frame}
\subsection{Przygotowanie Danych}
	\begin{frame}
		\frametitle{Przygotowanie Danych}
		Tu sprawdzić czy nie ma brakujących wartości i kodowanie
		
	\end{frame}
	\section{Model}
	\subsection{Modelowanie}
	\begin{frame}
		\frametitle{Modelowanie}
		Przetestować różne modele (KNN, lasy, bootsrap), na różnych wielkościach zbioru uczącego.
		
	\end{frame}
	\subsection{Ewaluacja}
\begin{frame}
	\frametitle{Ewaluacja}
	Porównanie modeli (podejrzewam, że KNN wypadnie lepiej niż lasy)
	
\end{frame}

	\section{Wdrożenie}
		\begin{frame}
			\frametitle{Wdrożenie -- Wnioski}
			Wnioski z eksploracji (tego jeszcze nie wiem :)
			
				\vspace{0.5cm}
			
			\centering\Large{Dziękuję za uwagę!}
			
		\end{frame}
		
\end{document}